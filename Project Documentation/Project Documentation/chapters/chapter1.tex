\let\textcircled=\pgftextcircled
\chapter{Problem Definition}
\label{chap:intro}

\initial{R}esurfacing after the tribulations of COVID-19, travel is becoming more prominent than ever as people from all around the world wish to explore and experience the world. Planning and organising holidays has always been time consuming, stressful and confusing. Traverse is a sophisticated travel application that utilises machine learning to enable the user to develop a personalised itinerary within a target budget; and discover and book different transport, activities, restaurants and hotels based off independent reviews and ratings. Complementing the rating system, which is not completed by all travellers, the artificial intelligence will build profiles of different businesses and the frequency and polarity of visits and use that to recommend different activities. The mobile application allows users to store photos within the holiday timeline on their profile and share their experiences with only friends or the public domain. The user can also get inspiration for holiday ideas from friends and family, or the wider community of individuals and groups who travel the world. \\

Through answering a small number of questions, such as the priorities of travelling (food, culture, adrenaline inducing experiences), the machine learning model will recommend activities associated with a certain destination to the user, allowing for discovery of new places easily and efficiently. Another issue that is prevalent when travelling is the struggle in trying to book accommodation and flights, and the dilemma of which one to book first. Traverse will solve this problem, through simultaneous booking, enabling the user to book both the flights and accommodation without stressing over availability. In conjunction to this, the artificial intelligence will ensure that their is no overlaps in the schedule or excess waiting time for transportation, such as connecting flights, so that the travelling experience will be seamless. Thus the machine learning will use the user preferences, budget, time frame and destination to efficiently build a personalised itinerary. \\

The mobile application will give the user the ability to use flight numbers to retrieve flight information from the International Air Transport Association (IATA) and the International Civil Aviation Organization (ICAO) API which can then be input into the itinerary so that all the updated times and locations of each event, for example delayed flights, are in one place. The application will also use bus, ferry and train timetables to specify in their schedule the different options the user can use to get between each location, highlighting the most time and cost effective. Therefore, the traditional approach of navigating through websites and convoluted booking sites to try and plan travel is significantly improved by Traverse, as all aspects of planning travel are located in one application, allowing smooth travel that will lead to new experiences and adventures.



