
\let\textcircled=\pgftextcircled
\chapter{Issues Relevant to Program}

\initial{B}lockchain is an open digital ledger technology that has the capability of significantly altering the way that people operate in organizations. Blockchain’s ethical issues for business catalyse from its three main promises: immutability, disintermediation, and automation. Immutability results in the permanency of a human past record and raises ethical issues such as privacy and transparency concerns. Disintermediation refers to horizontal decision-making and the numerosity of stakeholders in verifying outcomes which raises ethical issues related to accountability and equal opportunity. Automation refers to the self-executing features of coded agreements called smart contracts which raises issues related to the absence of human decision making and the inability for human intervention. The main ethical issue that will be discussed, however, is the inadvertent asymmetry of power that is being created, as large corporations have access to more bargaining power, information, and efficient transactions then smaller possibly local businesses. \\

In this sense, this business blockchain might enable transactions that are the product of force or possibly even fraudulent activity that would normally be prevented through institutions such as banks or government bodies. These mediating institutions would normally identify and constrain the misuse of markets by large corporations, however, by managing their assets and transactions through the business blockchain, this might be circumvented. This would enable different types of illegal and immoral transactions by facilitating transactions without intermediaries who can personally be held accountable for those transactions. \\

A case in point is the “assassination markets.” AUGUR is an Ethereum-based blockchain application for the creation of “peer-to-peer prediction markets” which allow people to place bets secretly. AUGUR has created, invertedly, a utilisation of blockchain analogous to the “assassination market” in that it allows anonymous betting upon someone's death, which in turn may incentivize people to kill others so as to win these very bets. \\

Although this real-world example is slightly disconnected from business blockchains the same unintended use cases of the software may occur. The actualisation of this ethical issue in this business blockchain may come in the form of businesses trading slave labour that breaks jurisdiction regarding minimum wages and the fair work rights, however, due to the transaction being mediated through blockchain technology, the unethical actions might go unnoticed.
